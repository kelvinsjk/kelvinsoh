\documentclass[a4paper,12pt]{exam}
\usepackage[theoremfont,trueslanted,largesc,p]{newpxtext}
\usepackage{amsmath, amssymb}
\usepackage{unicode-math}
\setmathfont{Asana-Math}
\usepackage{enumitem}
\usepackage{multicol}
\usepackage{microtype}
\usepackage{hyperref}
\usepackage{graphicx}
\renewcommand{\rmdefault}{put}
\usepackage{parskip}
\newlength{\currentparskip}
\hyphenpenalty=9999
\exhyphenpenalty=9999
%%%%%%%%%%%%%%%%%%%%%%%%%%%%%%%%%%%%%%%%%%%%%%%%%%%%%%%%%%
\pagestyle{headandfoot}
\firstpageheader{}{\textbf{Mathlified Demo}}{2023}
\runningheader{}{}{}
\headrule
\footer{Kelvin Soh}{\url{https://github.com/kelvinsjk}}{Page \thepage}
\footrule
\pointsinrightmargin
\bracketedpoints
\pointsdroppedatright

\begin{document}

\begin{questions}

\question[2]
	Simplify
    ${(3 x^3)^2 - (3 x - 4)^2.}$
\droppoints


\question[2]
	Express
    ${x^2 + 4 x - 8}$
    in the form ${(x+h)^2 + k.}$
\droppoints


\question[2]
	Solve
    ${3 x^2 + 19 x + 9 > 3.}$
\droppoints


\question[3]
	Solve
    $$\frac{ 8 x^2 - 13 x }{ x - 2 } - (5 x + 3) = -5.$$
\droppoints


\question[2]
	Long divide
    ${\displaystyle \frac{x^2 + 3 x - 4}{x + 8}.}$
\droppoints


\question
	
	\begin{parts}
		\part[1]
			Make ${x}$ the subject from the following equation:
    $$\ln (5 - x) = - 7 t.$$
		\droppoints

		\part[1]
			What can you say about ${x}$ for large values of ${t?}$
		\droppoints

	\end{parts}

\question
	
	\begin{parts}
		\part[3]
			Make ${y}$ the subject from the following equation:
    $$\ln \left( \frac{2+y}{2-y} \right) = 5 t.$$
		\droppoints

		\part[1]
			*** What can you say about ${y}$ for large values of ${t?}$
		\droppoints

	\end{parts}

\question
	Evaluate the following, leaving your answer in exact form.
	\begin{parts}
		\part[1]
			${\sin  \frac{\pi}{6},}$
		\droppoints

		\part[1]
			${\cos  \frac{\pi}{6},}$
		\droppoints

		\part[1]
			${\tan  \frac{\pi}{6},}$
		\droppoints

		\part[1]
			${\sin  \frac{\pi}{4},}$
		\droppoints

		\part[1]
			${\cos  \frac{3\pi}{4}.}$
		\droppoints

	\end{parts}

\question[3]
	Solve, for ${0 \leq \theta \leq \frac{\pi}{2},}$
    $$\cos 2 \theta  + 5 \sin \theta  - 3 = 0.$$
\droppoints


\question[3]
	Find the values of the constants ${a}$ and
    ${b}$ such that
    $$- \sec^2 \theta - 9 = a\tan^2 \theta + b$$
    for all values of ${\theta.}$
\droppoints



\end{questions}

\newpage

\section*{Answers}\begin{enumerate}
\item
	$9 x^6 - 9 x^2 + 24 x - 16.$

\item
	$\left( x + 2 \right)^2 - 12.$

\item
	${x < -6}$ or ${x > - \frac{1}{3}.}$

\item
	${x=-1}$ or ${x=\frac{4}{3}.}$

\item
	$\displaystyle x=x - 5 + \frac{36}{x + 8}.$

\item
	
	\begin{enumerate}
		\item
			$\displaystyle x = 5-\mathrm{e}^{- 7 t}.$
		\item
			${x \to 5.}$
	\end{enumerate}

\item
	
	\begin{enumerate}
		\item
			$\displaystyle x = \frac{2 \mathrm{e}^{5 t} - 2}{\mathrm{e}^{5 t} + 1}.$
		\item
			${y \to 2.}$
	\end{enumerate}

\item
	
	\begin{enumerate}
		\item
			${\frac{1}{2}.}$
		\item
			${\frac{1}{2} \sqrt{3}.}$
		\item
			${\frac{1}{3} \sqrt{3}.}$
		\item
			${\frac{1}{2} \sqrt{2}.}$
		\item
			${- \frac{1}{2} \sqrt{2}.}$
	\end{enumerate}

\item
	$\theta = \frac{1}{6} \pi.$

\item
	${a = -1,}$ ${b=-10.}$


\end{enumerate}

\end{document}